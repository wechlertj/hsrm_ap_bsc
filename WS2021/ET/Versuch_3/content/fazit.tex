\chapter{Fazit}
In diesem Bericht wurde eine einfache Emitterschaltung eines Transistors aufgebaut und auf ihre verschiedenen Kennlinien,Widerstände und Stromverstärkung untersucht.
Dabei wurde herausgefunden das der Basisstrom einer solchen Schaltung erst ab einem bestimmten Wert der Eingangsspannung steigt und vorher bei Null bleibt. Beim Eingangswiderstand sieht es anders aus, da dieser bei steigendem Basisstrom sinkt. Der Kollektorstrom dieser Emitterspannung steigt mit dem Basisstrom gemeinsam an, jedoch viel schneller. Zur bestimmung der Stromverstärkung wurden Messwerte aus LT-Spice übernommen. Dabei ist aufgefallen, das LT-Spice uns über Tausend Werte übermittelt hat. Es würde wahrscheinlich auch ausreichen weniger Werte zu entnehmen um die Stromverstärkung in Abhängigkeit des Kollektorstroms zu entnehmen. Eine größere Schrittweite hätte viel Arbeit gespart. Zum zeichnen des Graphen wurde dann von den ermittelten Verstärkungswerten und dem Kollektorstrom jeweils der Logarithmus genommen. Im Datenblatt des Bauteils BC547B ist der Graph jedoch anders skaliert. Es wäre ganz gut gewesen zu wissen wie man in Scidavis einen Graphen genauso skaliert. Beim differentiellen Ausgangswiderstand wurden die verschiedenen Messergebnisse in Abbildung 2.11 verbunden, um ein Gefühl für den Kurvenverlauf des Ausgangswiderstandes zu bekommen.