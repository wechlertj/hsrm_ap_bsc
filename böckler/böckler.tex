\documentclass[12pt]{scrreprt}
\usepackage{graphicx}
\usepackage{geometry}
\geometry{
 a4paper,
 left=20mm,
 right=20mm,
 top=25mm,
 headheight=20mm,
 headsep=0mm,
 bottom=30mm,
}
\usepackage[T1]{fontenc}				% Trennung von Umlauten.
\usepackage[ngerman]{babel}	
\usepackage[utf8]{inputenc}
\usepackage[headsepline]{scrlayer-scrpage}
\ModifyLayer[addvoffset=-10pt]{scrheadings.head.below.line}
\usepackage{blindtext}
\usepackage[font=small,labelformat=empty,justification=centering]{caption}
\setlength{\parindent}{0em}

\usepackage[version=4]{mhchem}

%Autordaten
\newcommand{\datum}{09.07.2020}
\newcommand{\autorinfo}{\textbf{Wechler, Tim-Jonas} (1137877)}


\ihead{\textbf{WT 1} \datum}
\chead{\autorinfo}
\ohead{\vspace{10pt}\includegraphics[width=4cm]{logo_simple}}
 
\begin{document}

\section*{Warum bewirbst du dich bei der Hans-Böckler-Stiftung?}
Ich bin auf die HBS durch einen Kommilitonen aufmerksam geworden, der selbst Stipendiat
bei Ihnen ist. Er hat mir sehr begeistert von den Workshops und Seminaren berichtet, die er
bei Ihnen absolviert hat. Des Weiteren habe ich gelesen, dass die Hans-Böckler-Stiftung hohen
Wert auf gesellschaftliches Engagement legt. Auch ich habe mich schon immer gerne engagiert.
Unter anderem war ich mehrere Jahre in der Jugendarbeit tätig und aktuell bin ich Mitglied
im Studierenden Parlament und studentischer Vertreter im Fachbereichsrat Ingenieurswissenschaften,
wo ich die Interessen meiner Kommiliton*innen vertrete.
\section*{Wie würdest du dein Studium dann finanzieren?}
Über BaföG und Minijobs, beziehungsweise meine Arbeit am Institut.
\section*{Welche besonderen Hindernisse musstest du überwinden?}
Die Trennung meiner Eltern war für mich die bisher größte Herausforderung. Ich habe durch
diese Trennung vieles über mich selbst gelernt und mich auf eine viel bewusstere Art und Weise
kennen lernen können. Des Weiteren waren die Vorbereitung und der Anfang des Studiums
eine große Herausforderung für mich. Ich hatte eine eigene Wohnung, ein Auto, regelmäßiges
Einkommen und wohnte in der Nähe meiner Familie. All diese Sicherheiten gab ich auf und
überwand meine eigenen Ängste und Zweifel, um mir meinen Traum, ein Studium durchzuführen,
zu erfüllen.
\section*{Wie und in welcher Form bist du unterstützt worden?}
Unterstützung bekomme ich aus meinem gesamten sozialen Umfeld. Angefangen bei meiner
Familie, Verwandten, meinem Freundeskreis bis hin zu meinen Kommiliton*innen und (ehemaligen)
Arbeitskolleg*innen. Die Form der Unterstützung hängt sehr stark davon ab in welcher
Situation ich mich befinde. Sie reicht von guten Ratschlägen bis hin zu konkreter Hilfestellung.
\section*{Warum sollten wir dich fördern?}
Die aktuelle wirtschaftliche wie gesellschaftliche Situation geht immer weiter in Richtung der
Quantifizierung allem Handeln und Sein. Was ich dabei beobachte ist, dass jegliche Menschlichkeit
und damit Gleichberechtigung verloren geht. Mit dieser Beobachtung beschäftige ich mich
schon des längeren und möchte gerne daran etwas ändern. In meinem bisherigen Engagement
lag mir viel daran die Personen, mit denen ich arbeite, alle gleich zu behandeln. Besonders
wichtig ist mir hierbei das jeder zu mir kommen kann und ich diese Person ungeachtet ihrer
Vorgeschichte bestmöglich unterstütze. Ich setze meine Fähigkeiten und mein Wissen zum
Wohle anderer ein.
Als ich das Leitbild der Hans-Böckler-Stiftung gelesen habe, habe ich genau diese Grundsätze
entnehmen können. Es würde mich freuen wenn ich mein Engagement bei der Hans-Böckler-
Stiftung mit einbringen könnte und zu dem Ziel der Gleichberechtigung mein Beitrag leisten
könnte.
\section*{Woran bist du politisch interessiert?}
Themen wie Nachhaltigkeit und Klimawandel haben für mich politisch einen hohen Stellenwert,
genauso wie die Gleichberechtigung aller Menschen.
\section*{Was möchtest du mit deinem Engagement in der Gesellschaft verändern?}
In den Bereichen Nachhaltigkeit und Klimaschutz möchte ich als gutes Vorbild voran gehen
und Leute in meinem Umfeld motivieren sich auch dafür einzusetzen. Denn wir alle leben auf
demselben Planeten. Ein weiteres Problem in unserer Gesellschaft ist, dass alles quantifiziert
wird. Am Ende interessieren nur noch die Leistungen oder die Zahlen eines Quartals. Alles wird
immer unpersönlicher und jeder wird zum Einzelkämpfer. Wir Menschen sind aber immer noch
Geschöpfe, die nur als Gemeinschaft überleben können. Diesen Grundsatz möchte ich, so weit
es mir möglich ist, wieder in der Gesellschaft etablieren.

Nur zusammen können wir (über-)LEBEN.
\section*{Was willst du uns noch mitteilen?}
Ich möchte an dieser Stelle nochmal auf die letzte Frage genauer eingehen. Durch die Quantifizierung
unseres Lebens wird nur noch nach dem „Was“ oder „Wie“ gefragt (Begriff: „Know-How“).
Was sich nur noch wenige fragen ist das „Warum“. Ich beobachte, dass dadurch nur kurz bis
mittelfristig gehandelt wird und sich keiner mehr über seine eigene Zeit hinaus Gedanken macht.
 Harald Lesch hat in seinem Buch „Die Menschheit schafft sich ab!“ genau
diese Problematik, dass man sich nur um das Know-How kümmert, angesprochen. Er bringt den
Begriff „Know-Why“ ins Gespräch und damit die Frage nach dem „Warum“. Außerdem
hat sich der amerikanische Autor und TED-Talk-Sprecher Simon Sinek sich mit derselben Thematik
in seinem Buch „Start with Why“ befasst. Simon Sinek hat es aus der gesellschaftlichen
und wirtschaftlichen Sicht betrachtet. Er kam zu dem Schluss, dass man mit dem „Warum“
große gesellschaftliche und wirtschaftliche Veränderungen herbeiführen kann. Er verweist hier
auf Martin Luther King oder auf den lang anhaltenden Erfolg von Apple.
Harald Lesch und Simon Sinek kommen zu dem Punkt dass es wichtiger ist zu wissen „Warum“.
Durch die Antwort auf das „Warum“ ergeben sich ganz neue Perspektiven und Möglichkeiten.
Mit dem Lesen der Bücher von Harald Lesch und Simon Sinek und durch diese Fragen die Sie
mir stellen, habe ich mich sehr intensiv mit meinem Warum auseinander gesetzt.

Mein „Warum“ ist:

Mein Umfeld und die Gesellschaft in der ich lebe, sind mir sehr wichtig. Da ich ohne sie 
nicht (über-)LEBEN kann, möchte ich mich zum Wohle anderer einsetzen und aus Interesse für andere aktiv sein.
\end{document}
