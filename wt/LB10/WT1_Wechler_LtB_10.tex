\documentclass[12pt]{scrreprt}
\usepackage{graphicx}
\usepackage{geometry}
\geometry{
 a4paper,
 left=20mm,
 right=20mm,
 top=25mm,
 headheight=20mm,
 headsep=0mm,
 bottom=30mm,
}
\usepackage[T1]{fontenc}				% Trennung von Umlauten.
\usepackage[ngerman]{babel}	
\usepackage[utf8]{inputenc}
\usepackage[headsepline]{scrlayer-scrpage}
\ModifyLayer[addvoffset=-10pt]{scrheadings.head.below.line}
\usepackage{blindtext}
\usepackage[font=small,labelformat=empty,justification=centering]{caption}
\setlength{\parindent}{0em}

\usepackage[version=4]{mhchem}

%Autordaten
\newcommand{\datum}{09.07.2020}
\newcommand{\autorinfo}{\textbf{Wechler, Tim-Jonas} (1137877)}


\ihead{\textbf{WT 1} \datum}
\chead{\autorinfo}
\ohead{\vspace{10pt}\includegraphics[width=4cm]{logo_simple}}
 
\begin{document}
In der letzten Vorlesung von Werkstofftechnik 1 wurden die Themen der Legierung und Normierung von Stählen behandelt.\par\medskip 

Wenn es um das Thema \textbf{Legierung} sich handelt unterscheidet man Grundlegend zwischen hochlegiertem und niederlegiertem Stahl.
\textbf{Hochlegierter} Stahl hat Sondereigenschaften wohin gegen der \textbf{niederlegierte} Stahl sich aus verschiedene Elemente zusammensetzt. Hierbei ist zu beachten das es sich mindestens um 3 Legierungselemente handelt.\par\smallskip

Ein \textbf{Legierungselement} kann die Phasenzustände stark beeinflussen und reduziert im Allgemeinen die Umwandlungsgeschwindigkeit. 
Es ist zu beachten das sich Legierungselemente gegen seitig nicht in ihren Eigenschaften aufsummieren. Viel mehr gibt es eine überproportionale Auswirkung der Eigenschaften. Es kann auch passieren das sich die Elemente in ihren Eigenschaften gegenseitig beeinflussen und schwächen. 
Um von einem Legierungselementen sprächen zu können muss ein gewisser Grenzwert überschritten sein. Ist dieser nicht überschritten, so redet man von einem Begleiter. 
Der Grenzwert ist abhängig vom Werkstoff um den es geht. Im Allgemeinen liegt der Grenzwert bei einem prozentualen Anteil von $0,05\%$. Es gibt hier noch einige Stoffe die einen Abweichenden Wert haben. 
Dieser Grenzwert kann von $1,6\%$ bei Mangang bis $0,0008\%$ bei Bor gehen. Des weiteren sind  Kohlenstoff, Phosphor, Schwefel, Stickstoff und Sauerstoff keine Legierungselemente.

Bei der \textbf{Normierung} wurden uns das Thema in Form von mehreren Beispielen näher gebracht. Seit 1992 gibt es eine DIN-Norm (DIN EN 10 020) nach dem Stahl gekennzeichnet wird. 


\end{document}
