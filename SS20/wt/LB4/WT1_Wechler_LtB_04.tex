\documentclass[12pt]{scrreprt}
\usepackage{graphicx}
\usepackage{geometry}
\geometry{
 a4paper,
 total={210mm,297mm},
 left=20mm,
 right=20mm,
 top=30mm,
 headheight=10mm,
 bottom=30mm,
}
\usepackage[T1]{fontenc}				% Trennung von Umlauten.
\usepackage[utf8]{inputenc}				% Wird für die direkte Eingabe von Umlauten gebraucht.
\usepackage[ngerman]{babel}	
\usepackage[utf8]{inputenc}
\usepackage{fancyhdr}
\usepackage{blindtext}
\setlength{\parindent}{0em}

%Autordaten
\newcommand{\datum}{18.06.2020}
\newcommand{\autorinfo}{\textbf{Wechler, Tim-Jonas} (1137877)}

\pagestyle{fancy}
\fancyhf{}
\pagestyle{fancy}
\lhead{\textbf{WT 1} \datum}
\chead{\autorinfo}
\rhead{\includegraphics[width=4cm]{logo_simple}}
 
\begin{document}
In der letzten VL haben wir die Phasenumwandlung im festen Zustand und im Anschluss die Zustandsdiagramme besprochen.\\
Es gibt drei wesentliche Arten der Umwandlung. Einmal wird die Umwandlung durch eine Änderung der Struktur vollzogen, dann durch eine Konzentrationsänderung oder durch Änderung der Struktur und Konzentration.\\[0,3cm]
Bei der Änderung der Struktur gibt es nocheinmal einen unterscheidung ob der Werkstoff schnell und stark abgekühlt wird oder eine im Verhältnis eher langsamere abkühlung des Werkstoffs.\\
Wird schnell abgekühlt so spricht man von der Martensitische-Umwandlung. Hier werden C-Atome in einem Fe-Gitter (krz) eingeschlossen und erzeugen somit Spannungen um Gitter.
Passiert die abkühlung hingehen langsamer können die Atome ihre Plätze wechsel, man spricht hier von einer Polymorphen-Umwandlung. Dadurch das Atome ihre Plätze tauschen gibt es eine veränderung der Gitter-Struktur.
Diese Veränderung durchläuft ein Wechsel vom krz-Gitter zum kfz-Gitter wieder hin zum krz-Gitter. Abgeleitet von der änderung der Gitterstruktur kann man aussagen, dass sich das Volumen und die Dichte des Werkstoffs ändern.\\[0,2cm]
Die Umwandlung bei der sich die Konzentration ändert tritt dann auf wenn eine Legierung übersättigt ist. Hier sammeln sich die Atome und gruppieren sich zu einem Korn. Dies geschieht bei schneller abkühlung so mit kann isch kein gleichgewicht in der Legierung bilden.\\[0,2cm]
Die letzte Umwandlungsart beeinflusst die Struktur wie die Konzentration. Hier gibt es auch wieder zwei wesentliche unterschiede, die kontinuierliche und diskontinuierliche Ausscheidung.\\
Die kontinuierlichen Ausscheidung verläuft in zwei Schritten. Bei dem ersten Schritt wird die legierung soweit erhitzt das sich alle Atome gleichmäßig verteilen können. Danach wird der Werkstoff schnell abgekühlt damit keine Körner sich bilden können.
Im nächsten Schritt wird die Legierung ausgelagert. Für jede Legieung gibt bei charakteristischen Temperatur für die Auslagerung. Während der Auslagerung können Atome kleine Strecken zurücklegen. Die Kornbildung findet in feinen Strukturen statt. \\
Bei der diskontinuierlichen Ausscheidung wurde es anhand vom Eutektoiden Zerfalls erklärt. Allgemein kann man sagen wenn die Legierung abkühlt bei einer gewissen Temperatur eine übersättigung erreicht und sich eine eisenreiche und kohlenstoffreiche Phase bildet.\\[0,2cm]
Als letztes kamen die Zustandsdiagramme dran und hier wurde gerklärt wie man sie lesen kann und was ihre bedeutung ist. Der Umstand das sie nur über Versuche ermittelt wurden fande ich sehr faszinierend. Ich habe von Prof. Dr. X. Jiang, Uni Siegen (Chair of Surfaceand MaterialsTechnology) aus dem Jahr 2013 ein PDF gefunden in dem das Zustandsdigramm sehr ausfühlrlich behandelt wird \\
(Quelle: https://www.mb.uni-siegen.de/lot/studium/lehrveranstaltungen/wt1/\\wt1-6-2012w.pdf).\\
Was ich hier nochmals sehr gut fand war die Aussage das bei allen etzt beschrieben Prozessen die Atome ein stabilen Zustand mit wenig Energie bedarf suchen. Somit ergeben sich durch unterschiedliche Beahndlung des Werkstoffs unterschiedliche Phasenumwandlungen und letzendlich unterschiedliche strukturelle Zusammensetzungen. In diesem PDF gab es zum Schluss noch eine Darstellung eines Zustandsdiagramm mit drei Legierungsanteilen.\\[0,3cm]
Da der Platz demnächst aufgebraucht ist, mein Fazit über diese VL. Es war sehr aufschlussreich über die unterschiedlichen Umwandlungen zu hören und diese näher kennen zu lernen. Wie schon erwähnt fande ich die Zustandsdiagramme besonders interessant da man eine grafische Darstellung hatte die einem einen grobe Übersicht über das Verhalten der Legierung verschafft. In der PDF gab es noch weitere Bsp. die einem auch noch weitere Besonderheiten näher brachten.
 
\end{document}

